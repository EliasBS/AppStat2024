\documentclass[11pt]{article}

\usepackage[latin1]{inputenc}
\usepackage[danish]{babel}        % Use English headings, date format.
\usepackage{a4wide}               % A4 (DIN format).
\usepackage[hidelinks]{hyperref}  % Enable direct links in PDF (e.g. for data sets)

\textheight=1.03\textheight
\textwidth=1.10\textwidth
\hoffset=-0.05\textwidth
\leftmargin=-0.18\textwidth
\headsep=0.0pt
\headheight=0.0pt

\vfuzz2pt   % Don't report over-full v-boxes if over-edge is small
\hfuzz10pt  % Don't report over-full h-boxes if over-edge is smallish

\newcommand{\half}{\mbox{$\frac{1}{2}$}}

\begin{document}
%\pagestyle{empty}

%----------------------------------------------------------------------------
\noindent
University of Copenhagen \hfill
Niels Bohr Institute, \today \par
\vspace{-2ex}
\noindent
\hrulefill

\vspace{1ex}
\begin{center}
{\bf {\Huge Applied Statistics}}\\
\vspace{1ex}
{\large Exam in Applied Statistics 2022/23}
\end{center}

%----------------------------------------------------------------------------
\vspace{0ex}
\noindent
This take-home exam was distributed Thursday the 19th of January 2023 at 08:00. A solution in PDF format must be submitted at \texttt{\bf www.eksamen.ku.dk by 20:00 Friday the 20th}, along with all code used to work out your solutions (as appendix). Links to data files can also be found on the course webpage and github. Working in groups or discussing the problems with others is {\bf NOT} allowed.

\vspace{-1ex}
\begin{center}
  Thank you for all your hard work, Kate, Rajeeb, Ting-Yi, Emma, Malthe, Mathias, \& Troels.
\end{center}

%----------------------------------------------------------------------------

\noindent
\hrulefill\\
\emph{Science may be described as the art of systematic oversimplification.}\\
  \phantom{foobar} \hfill [Karl Popper, Austrian/British philosopher 1902-1994]\\[-2ex]

  
%----------------------------------------------------------------------------
\vspace{-2ex}
\noindent
\hrulefill

\vspace{4ex}
\noindent
{\bf I -- Distributions and probabilities:}
\begin{description}
  \item[1.1] (8 points)
  The scores of two tests (A \& B) are both Gaussianly distributed with $\mu=50$, $\sigma=20$.
  \vspace*{-1ex}
  \begin{itemize}
    \item What fraction of students will get a score in test A in the range [55,65]?
    \item What uncertainty on the mean score do you obtain from 120 B test scores?
    \item If scores correlate with $\rho_{A,B} = 0.60$, what fraction should get a score above 60 in both tests?
  \end{itemize}
%
  \item[1.2] (4 points)
  A game is designed such that on average 40\% of persons will win it.
  \vspace*{-1ex}
  \begin{itemize}
    \item If 20 random persons play the game, what is the chance that 11 or more will win it?
  \end{itemize}
\end{description}



%----------------------------------------------------------------------------

\vspace{2ex}
\noindent
{\bf II -- Error propagation:}
\begin{description}
\item[2.1] (8 points)
  Let $x = 1.033 \pm 0.014$ and $y = 0.07 \pm 0.23$, and let $z_1 = xy e^{-y}$ and $z_2 = (y+1)^{3}/(x \!-\! 1)$.
  \vspace*{-1ex}
  \begin{itemize}
    \item Which of the (uncorrelated) variables $x$ and $y$ contributes most to the uncertainty on $z_1$?
    \item What are the uncertainties of $z_1$ and $z_2$, if $x$ and $y$ are correlated with $\rho = 0.4$?
    \item What is the Pearson correlation between $z_1$ and $z_2$ for $z_1 \in [-1,1]$ and $z_2 \in [-1,1]$?
  \end{itemize}
%
\item[2.2] (7 points)
  In a (Cavendish) experiment, you have made five measurements of Earth's density $\rho$:\\[-1ex]
  \vspace*{-3ex}
  \begin{center}
  \begin{tabular}{lccccc}
    % \hline
    \hline
    Observation                   &1                &2                 &3                &4                &5\\
    Result (in g/$\mbox{cm}^3$)   &5.50 $\pm$ 0.10  &5.61 $\pm$ 0.21   &4.88 $\pm$ 0.15  &5.07 $\pm$ 0.14  &5.26 $\pm$ 0.13\\
    % \hline
    \hline
  \end{tabular}
  \end{center}
% Cavendish original data (though missing one entry!):
%   5.50 5.61 4.88 5.07 5.26 5.55 5.36 5.29 5.58 5.65 5.51 5.57 5.53 5.62 5.29 5.44
%   5.34 5.79 5.10 5.27 5.39 5.42 5.47 5.63 5.34 5.46 5.30 5.75 5.68 5.85
  \vspace*{-3ex}
  \begin{itemize}
    \item What is the combined result and uncertainty of these five measurements?
    \item Are your measurements consistent with each other? If not, what is then your best estimate?
    \item The precise value is $5.514$ g/$\mbox{cm}^3$. How consistent is you measurement with this number?
  \end{itemize}
%
\item[2.3] (7 points)
  An ellipse $E$ has semi-major axis $a = 1.04 \pm 0.27$ and eccentricity $e = 0.71 \pm 0.12$.
  \vspace*{-1ex}
  \begin{itemize}
    \item The area $A$ of an ellipse is generally $A = \pi a^2 \sqrt{1 - e^2}$. What is the area of the ellipse $E$?
    \item The circumference $C$ has no formula but can be bounded as $4a \sqrt{2-e^2} < C < \pi a \sqrt{4-2e^2}$.
      What value and uncertainty for $C$ would you give?
  \end{itemize}
\end{description}



%----------------------------------------------------------------------------

% \newpage
\noindent
{\bf III -- Simulation / Monte Carlo:}
\begin{description}
  \item[3.1] (8 points)
    You are optimising container transport, in particular the time, $\Delta t$, between the
    daily truck arrivals (120 minutes uncertainty) and the ship departure (50 minutes uncertainty).
  \vspace*{-1ex}
  \begin{itemize}
    \item If $\Delta t$ = 130 minutes, what fraction of containers will have to wait to the next day?
    \item For what value of $\Delta t$ do containers, on average, have the least waiting time?
  \end{itemize}
%
  \item[3.2] (13 points)
    The Rayleigh distribution is a PDF given by: $f(x) = \frac{x}{\sigma^2} \exp(-\frac{1}{2}x^2/\sigma^2)$,
    with $x \in [0,\infty]$.
  \vspace*{-4ex}
  \begin{itemize}
    \item By what method(s) would you generate random numbers (from uniform) according to $f(x)$?
    \item Generate N=1000 random numbers according to $f(x)$ for $\sigma = 2$, and plot these.
    \item Fit this distribution of random numbers. How well can you determine $\sigma$ from the fit?
    \item Test the $1/\sqrt{N}$ scaling of the $\sigma$ fit uncertainty for $N \in [50,5000]$.
  \end{itemize}
\end{description}


%----------------------------------------------------------------------------

\noindent
{\bf IV -- Statistical tests:}
\begin{description}
\item[4.1] (15 points)
  Patients are either healthy or infected with Anoroc disease and their temperature, blood pressure and age is found in
  \href{http://www.nbi.dk/~petersen/data\_AnorocDisease.txt}{\bf www.nbi.dk/$\sim$petersen/data\_AnorocDisease.csv}.
  For patients 1-800 (control) the outcome in known, while it is unknown for patients 801-1000 (unknown).
  \vspace*{-1ex}
  \begin{itemize}
    \item Using the control sample, plot the three distributions for healthy and sick, respectively.
      Which of the three single measures gives the highest separation between healthy and sick?
    \item Test if the age distribution is statistically the same between healthy and sick.
    \item Given any combination of all three variables, separate the two groups as well as possible
      and estimate the number of infected patients in the unknown group.
    \item Assuming a prior probability of $p=0.01$ of being ill, what is the probability that a new patient with
      $T = 38.6~ \mbox{C}^{\circ}$ is ill?
  \end{itemize}
  %
\item[4.2] (14 points)
  The file \href{http://www.nbi.dk/~petersen/data\_CountryScores.txt}{\bf www.nbi.dk/$\sim$petersen/data\_CountryScores.csv}
  contains a list of countries along with several key numbers and indices.
  \vspace*{-1ex}
  \begin{itemize}
    \item Determine the mean, median, 25\%, and 75\% quantiles of the GDP.
    \item Does the distribution of $\log_{10}$(PopSize) follow a Gaussian distribution?
    \item What are the Pearson and Spearman correlations between happiness and education indeces?
    \item Plot the Happiness-Index as a function of GDP, and fit the relation between the two.
      From this fit, what would you estimate the uncertainty to be on the Happiness-index?
  \end{itemize}
\end{description}


%----------------------------------------------------------------------------

\noindent
{\bf V -- Fitting data:}
\begin{description}
\item[5.1] (16 points)
  The file
  \href{http://www.nbi.dk/~petersen/data\_GlacierSizes.txt}{\bf www.nbi.dk/$\sim$petersen/data\_GlacierSizes.csv}
  contains the estimated area and volume including uncertainties of 434 glaciers with an area above 1 $\mbox{km}^2$.
  \vspace{-1.0ex}
  \begin{itemize}
    \item Plot volume as a function of area. Which of the two have largest relative uncertainties?
    \item Fit data with the expected Area-Volume relation $V \sim A^{3/2}$. Assume no area uncertainties.
    \item Are you satisfied with the fit? And if not, point out its specific deficiencies.
    \item Fit again with improved functional form(s), and quantify the improvements.
    \item Redo this fit including the uncertainties in area. How large is the effect of including these?
    \item What volume and with what uncertainty would you expect a glacier of area 0.5 $\mbox{km}^2$ to have?
  \end{itemize}
\end{description}

\noindent
\hrulefill\\
\emph{Don't worry too much about statistics! Just tell us what you do, and do what you tell us.}\\
  \phantom{foobar} \hfill [Roger Barlow, ICHEP conference 2006, Moscow]\\[-2ex]


%----------------------------------------------------------------------------


\end{document}

%%% Local Variables: 
%%% mode: latex
%%% TeX-master: t
%%% End: 

